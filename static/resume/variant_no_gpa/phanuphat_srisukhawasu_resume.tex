\documentclass{phanuphat_srisukhawasu_resume} % Use the custom phanuphat_srisukhawasu_resume.cls style

\usepackage[left=0.4 in,top=0.4in,right=0.4 in,bottom=0.4in]{geometry} % Document margins
\usepackage[utf8]{inputenc}
\usepackage[default]{lato}
\DeclareRobustCommand\FAone{\fontencoding{U}\fontfamily{fontawesomeOne}\selectfont}
\newcommand{\tab}[1]{\hspace{.2667\textwidth}\rlap{#1}} 
\newcommand{\itab}[1]{\hspace{0em}\rlap{#1}}

%----------------------------------------------------------------------------------------
% Contact Information
%----------------------------------------------------------------------------------------

\name{Phanuphat Srisukhawasu} % Your name
\address{+65 8090 5619 \\ 
  Singapore \\
  \href{mailto:phanuphat.srisukhawasu@gmail.com}{phanuphat.srisukhawasu@gmail.com}
} 
\address{\href{https://github.com/oadultradeepfield}{github.com/oadultradeepfield} \\
  \href{https://linkedin.com/in/phanuphats}{linkedin.com/in/phanuphats} \\
  \href{https://phanuphats.com}{phanuphats.com}
} 

\begin{document}

%----------------------------------------------------------------------------------------
% Professional Summary
%----------------------------------------------------------------------------------------

% \begin{rSection}{Professional Summary}
%
%   {\textbf{Astronomy Olympiad medalist} turned Computer Science major with a \textbf{focus on measurable outcomes} and keen attention to detail. Passionate about building \textbf{scalable systems} and applying technology to create positive impact across disciplines.}  
%
% \end{rSection}

%----------------------------------------------------------------------------------------
% Education
%----------------------------------------------------------------------------------------

\begin{rSection}{Education}

  \textbf{Bachelor of Computing, Computer Science, National University of Singapore} \hfill {Expected Graduation: Dec 2026}
  \begin{itemize}
    \item GPA:\@ First Class Honours
    \item Notable Coursework: Design \& Analysis of Algorithms, Operating Systems, Programming Languages, Software Engineering
  \end{itemize}

\end{rSection}

%----------------------------------------------------------------------------------------
% Work Experience
%----------------------------------------------------------------------------------------

\begin{rSection}{Work Experience}

  \textbf{Software Engineer Intern} \hfill May 2025 --- Aug 2025 \\
  Computing for Voluntary Welfare Organisations (CVWO) \hfill \textit{Singapore}
  \begin{itemize}
    \item \textbf{Collaborated with a 10-person team} to design web (React, Redux, Go, Ruby, Rails, PostgreSQL), PWA (Ionic), and mobile (React Native) applications for Active Ageing Centre operations as part of a \textbf{GIC-supported impact program.}
    \item \textbf{Migrated legacy Ruby on Rails financial reporting module to modern React, Ionic, and Go stack}, independently delivering \textbf{22,000+ insertions and 12,000+ deletions} while upgrading CI/CD pipelines to improve maintainability.
    \item \textbf{Rewrote account statement querying logic} using concurrent goroutines, \textbf{overcoming critical performance bottlenecks} for processes handling 4,000+ transactions across 800+ client accounts, scheduled monthly via \textbf{cron job.}
    \item \textbf{Engineered reusable digital consent management system} in Go and React Native, enabling hospitals to contact patients' next of kin with e-signatures, \textbf{eliminating manual processes for 2,000+ patients.}
  \end{itemize}

  \textbf{Research Student} \hfill May 2022 --- Mar 2024 \\
  Department of Physics, Faculty of Science, Ramkhamhaeng University \hfill \textit{Bangkok, Thailand}
  \begin{itemize}
    \item Examined deep learning solutions for astronomy and space physics problems under Dr. Suttiwat Madlee's guidance.   
    \item Developed an automated cosmic ray detection system for small cloud chambers using \textbf{YOLO models in PyTorch}, reducing analysis time from \textbf{hours to minutes} while maintaining \textbf{80\% accuracy} (\href{https://iopscience.iop.org/article/10.1088/1742-6596/2653/1/012007}{doi.org/10.1088/1742-6596/2653/1/012007}).
    \item Innovated a \textbf{novel X-class solar flare identification} method utilizing custom convolutional autoencoders, achieving \textbf{30\% accuracy improvement} and resolving class imbalance issues (\href{https://essopenarchive.org/users/803138/articles/1189366-enhancing-strong-solar-flare-prediction-using-convolutional-autoencoders-for-anomaly-detection-on-hmi-magnetograms}{doi.org/10.22541/essoar.174431882.20472576/v1}).
  \end{itemize}

\end{rSection} 

%----------------------------------------------------------------------------------------
% Projects
%----------------------------------------------------------------------------------------

\begin{rSection}{Projects}

  \textbf{Planetary Image Stacker} (\href{https://github.com/oadultradeepfield/planetary-image-stacker}{github.com/oadultradeepfield/planetary-image-stacker})
  \begin{itemize}
    \item Engineered a \textbf{fast, memory-efficient C++ planetary image stacking tool with OpenCV}, consolidating the core functionality of two popular software \textbf{(PIPP and AutoStakkert!)} into a single streamlined application.
    \item Optimized \textbf{image alignment and stacking algorithms via OpenMP parallelization}, delivering a 3.5x faster performance.
  \end{itemize}

  \textbf{RedactKit - TikTok TechJam 2025} (\href{https://devpost.com/software/redactkit}{https://devpost.com/software/redactkit})
  \begin{itemize}
    \item Built an \textbf{on-device machine learning model} for PII redaction within iOS devices, leveraging \textbf{Swift, PyTorch and CoreML.}
    \item Designed and \textbf{fine-tuned a NeuroBERT-Mini} pipeline, integrating the \textbf{OpenAI API} for PII placeholder generation and applying \textbf{backward fake-PII injection} to synthesize training data, achieving a \textbf{95\% F1-score} on a custom test set.  
  \end{itemize}

  \textbf{Boonchubike CMS} (\href{https://phanuphats.com/projects/boonchubike-cms}{phanuphats.com/projects/boonchubike-cms} $|$ \href{https://github.com/oadultradeepfield/thai-address-api}{github.com/oadultradeepfield/thai-address-api})
  \begin{itemize}
    \item Built a \textbf{client management system web app} for a bicycle business in Thailand using \textbf{React, TanStack, and Firebase.}
    \item Implemented a \textbf{reusable PDF generation feature} integrating the \textbf{Go-based Thai Address API} to automate delivery labels, \textbf{eliminating 800+ pages} of manual Word documents and \textbf{improving reliability.}
  \end{itemize}

\end{rSection}


%----------------------------------------------------------------------------------------
% Publications
%----------------------------------------------------------------------------------------

% \begin{rSection}{Publications}
%
%   \begin{itemize}
%     \item \textbf{Srisukhawasu, P.}, Silapasart, A., Limjanon, T., Madlee, S., Samanrak, C., Somnam, T. \textbf{DeepHCC:\@ Deep learning model for real-time count rate determination of particles in homemade cloud chamber.} Journal of Physics: Conference Series, vol. 2653, no. 1, 2023\@. \href{https://doi.org/10.1088/1742-6596/2653/1/012007}{doi.org/10.1088/1742-6596/2653/1/012007}.
%     \item \textbf{Srisukhawasu, P.}, Somnam, T., Madlee, S. \textbf{Enhancing Strong Solar Flare Prediction Using Convolutional  Autoencoders for Anomaly Detection on HMI Magnetograms.} ESS Open Archive, 2025\@. \href{https://doi.org/10.22541/essoar.174431882.20472576/v1}{doi.org/10.22541/essoar.174431882.20472576/v1}.
%   \end{itemize}
%
% \end{rSection}
%
%----------------------------------------------------------------------------------------
% Skills
%----------------------------------------------------------------------------------------

\begin{rSection}{Skills}

  \begin{tabular}{@{} >{\bfseries}l @{\hspace{6ex}} l}
    Programming Languages & Go, Java, Python, C++, C, SQL, JavaScript, TypeScript, Ruby, R, HTML, CSS \\ 
    Frameworks, Libraries, \& Tools & React, Next.js, Tailwind CSS, Rails, PostgreSQL, Docker, AWS, GCP, Git, PyTorch
  \end{tabular}

\end{rSection}

%----------------------------------------------------------------------------------------
% Achievements
%----------------------------------------------------------------------------------------

\begin{rSection}{Achievements}

  \begin{itemize}
    \item \textbf{International Astronomy Olympiad (IAO) 2021} --- Silver Medal (Top 8\%) 
    \item \textbf{International Olympiad on Astronomy and Astrophysics (IOAA) 2022/23} --- Bronze Medal
  \end{itemize}

\end{rSection}

\end{document}
